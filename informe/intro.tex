\chapter{Introducción}

Brainstorm!


*algo historia de la consola, que fue la más exitosa de su época, como trascendio, que era barata,etc*

*porque la elegimos: es una de las consolas mas investigadas y toqueteadas que alguna vez existio.*

*porque queremos independizarnos del sistema operativo: para poder correr nuestro emulador en máquinas mas restringidas, para que en un futuro sea posible correrlo en procesadores de bajo poder, se podría hacer un clon de la consola a muy bajo precio, un arduino cuesta 10~20 dolares. en marcado libre las consolas originales están como 1500 pesos :o*

*modo de trabajo: se busca no reinventar la rueda, el projecto se construyo tomando como guia otros kernels y emuladores ya existentes**explicar como se encontró y utilizo la información necesaria para hacer el proyecto*

*la mayoría se va a hacer en c, solo se usa assembler si es necesario*
*se asume conocimientos en c por parte del lector y un manejo de inglés*
*comentar que se van a usar referencias a los códigos del emulador*

\section{Estructuración}

En un principio dividimos en dos partes principales el projecto.

Una es el \textbf{emulador} propiamente dicho, es decir la que se encarga de hacer todo lo que hacía internamente la consola.

La otra parte es el \textbf{kernel} encargada de inicializar todo lo necesario para que el emulador funcione, así como proveerle funciones de bajo nivel tales como escribir en pantalla o reservar memoria. Al no tener un sistema operativo detrás, funciones como malloc y free que cualquier programador de C supone siempre presentes deben ser implementadas por el kernel.
